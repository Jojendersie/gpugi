\documentclass[english,10pt,a4paper,twocolumn,colorscheme=green]{orarticle}
\usepackage{hyperref}

\usepackage{listings}
\lstset{
	language=C++,
	commentstyle=\color{darkgreen}\textit,
	frame=single,
	tabsize=4,
	keywordstyle=\color{blue}\bfseries,
	stringstyle=\color[gray]{0.4},
	basicstyle=\lst@ifdisplaystyle\scriptsize\else\ttfamily\fi,
	morekeywords={uint16, uint32, Vec2, Vec3}
}

\begin{document}

	\titlehead{\begin{minipage}{0.5\linewidth}
		\begin{center}
		Otto-von-Guericke-University Magdeburg\\
		Faculty of Computer Science\\
		ISG - Computational Visualistics
		\end{center}
	\end{minipage}}
%	\subject{File Format Documentation}
	\title{The *.rawscene Format from GPUGI Project\\Version 1.0}
	\authors{Johannes Jendersie}
	\abstract{Abstract}{
		This is a complete overview about the contents of *.rawscene format. The file contains a list of different arrays. Each array has a name where some must exist in the file and some are optional.
	}
	\thispagestyle{empty}
	\maketitle
	
	
	
	% ************************************************************************ %
	\section{General Array Chunks}
	
	The file contains an arbitrary number of structured arrays. Each is introduced by the following header.
	
	\begin{lstlisting}
struct
{
	char name[32];
	uint32 numElements;
	uint32 elementSize;
};
	\end{lstlisting}
	
	The \lstinline|name| defines what kind of data is to be expected. The names are always lower case. A name and the associated structure should not change. Certain chunks are guaranteed to be existent e.g. "vertices". The optional ones are marked with surrounding [] in the title.
	
	If the loader does not support one data chunk it can be skipped. The number of bytes to the next header can be determined by \lstinline|numElements * elementSize|.
	
	% ************************************************************************ %
	\subsection{Vertices}
	\lstinline|header.name == "vertices"|
	
	The data for all geometry. There is exact one such array. Different objects
	or materials are not distinguished here.
	\begin{lstlisting}
struct Vertex
{
	Vec3 position;
	Vec3 normal;
	Vec2 texcoord;
};
	\end{lstlisting}
	
	% ************************************************************************ %
	\subsection{Triangles}
	\lstinline|header.name == "triangles"|
	
	Kind of an index buffer. Three succeeding indices form a triangle. There may be invalid triangles with three equal indices (all 0). These are added for padding purposes of the hierarchy assignment.
	\begin{lstlisting}
struct Triangle
{
	uint32 indices[3];
};
	\end{lstlisting}
	The \lstinline|elementSize| in the header is not that of a Triangle. It is a multiple of \lstinline|sizeof(Triangle)| and denotes the block size for hierarchies. The mentioned padding is used to fill the blocks. So once an invalid triangle occurs all following till the next block will be invalid too.
	
	% ************************************************************************ %
	\subsection{Material}
	\lstinline|header.name == "materialref"|\\
	\lstinline|header.name == "materialassociation"|
	
	"materialref" contains on entry for each material. The stored names reference material descriptors in an additional plain text file.
	
	"materialassociation" has one entry for each triangle and represents an index into the "materialref" array.
	\begin{lstlisting}
struct MaterialRef
{
	char material[32];
};

struct MaterialAssociation
{
	uint16 materialID;
};
	\end{lstlisting}
	
	% ************************************************************************ %
	\subsection{Hierarchy [Optional]}
	\lstinline|header.name == "hierarchy"|
	
	A potential n-ary tree. To iterate over all children of a node go to the \lstinline|firstChild| and follow the \lstinline|escape| pointer as long as the parent is equal. To iterate through the whole tree follow the \lstinline|firstChild| to a leaf (see next note) and each time you reach a leaf go to \lstinline|escape|.
	
	The first bit (most significant bit) of \lstinline|firstChild| denotes if the child is a leaf. In case it is set the remaining 31 bit are a block index in the "triangles" buffer. One leaf can contain more than one triangle which is encoded in the \lstinline|elementSize| of the "triangles" header. Unused triangles are filled with invalid indices (all three 0).
	\begin{lstlisting}
struct Node
{
	uint32 parent;
	uint32 firstChild;
	uint32 escape;
};
	\end{lstlisting}
	
	% ************************************************************************ %
	\subsection{Bounding Geometry [Depends on Hierarchy]}
	\lstinline|header.name == "bounding_aabox"|\\
	\lstinline|header.name == "bounding_sphere"|
	
	Exact one geometry array must exist if "hierarchy" exists. The number of elements in "hierarchy" and this array must be equal. There are different types of bounding volumes for the hierarchy nodes.
	\begin{lstlisting}
struct Sphere
{
	Vec3 center;
	float radius;
};

struct Box
{
	Vec3 min;
	Vec3 max;
};
	\end{lstlisting}
	
	
		
	% ************************************************************************ %
	\section{Material File}
	Materials are not stored in the binary format to make them better exchangeable. They are defined in a second file with the same name as the scene file, but with *.json ending.
	\begin{lstlisting}
{
  	"ExampleMaterial": 
  	{
		"diffuse": [0.5, 0.5, 0.5],
		"reflectiveness": [1.0, 1.0, 1.0, 5.0],
		"refractionIndexN": [1.3, 1.0, 0.75],
		"refractionIndexK": [3.0, 3.3, 3.9],
		"opacity": [1.0, 1.0, 1.0]
  	},		
	... more materials
}
	\end{lstlisting}
	Each data except \lstinline|"refractionIndexN"| and \lstinline|"refractionIndexK"| comes in two variants: One without \lstinline|"Tex"| suffix (must be a RGB constant) and one with this suffix containing (e.g. \lstinline|"diffuseTex"|) a file name of a RGB texture. One of both must exist. If both are given only the texture is used.
	
	\subsection{Intended Interpretation}
	The light is divided into components of different reflection pattern.
	
	\lstinline|"diffuse"|: Lambert-emitter and absorption. Of all light which is not reflected by the other components (1-\lstinline|"diffuse"|) is absorbed and the remainder diffuse reflected.
	
	\lstinline|"reflectiveness"|: Maximal amplitude (percentage) of Fresnel-reflected light. Think of a multilayer model. The upper layer is a perfect Fresnel-microfacet reflector. The actual reflection/transmission also depends on the angle.
	
	The fourth component is the exponent or sharpness of the specular reflection/refraction. Can also be interpreted as (1/roughness). Controls the shape of the Fresnel reflected/refracted lobe.
	
	\lstinline|"refractionIndex"|: Controls the Fresnel-function. This cannot be a texture.
	
	\lstinline|"opacity"|: From all light which passed the Fresnel surface the opaque part is passed to the diffuse term. The remainder is refracted with a cosine lobe. %Physically this is something like $scattering/dx$. $e^{-d \cdot (1/(1-opacity)-1)}$ is refracted, where $d$ is the length of the path through the material, and the remainder is the part which remains for diffuse/absorption.
	
	\subsection{Example Refraction Indices}
	For most materials you need to set the complex refraction indices $N = n + i \cdot k$. To get a feeling for the behavior look at the \href{	http://de.wikipedia.org/wiki/Fresnelsche\_Formeln\#mediaviewer/File:Fresnel\_reflection\_coefficients\_\%28DE\%29.svg}{picture from wikipedia}. Both coefficients are given for 3 RGB related wavelengths: R-610nm, G-546nm, B-436nm. I found no real source which maps pure RGB colors to wavelengths. The factors are derived from a multitude of RGB curve-pictures.
	
	There is an online database for refraction indices: http://refractiveindex.info/.
	From this here are some example materials.
	
	\begin{tabular}{|r|l|l|l|}
		\hline
		& R-610nm & G-546nm & B-436nm\\
		\hline
		Water n & 1.3320 & 1.3332 & 1.3376\\
		k & 1.2100e-8 & 1.8576e-9 & 1.1768e-9\\
		\hline
		Ethanol n & 1.3610 & 1.3631 & 1.3693\\
		k & 0.0000 & 0.0000 & 0.0000\\
		\hline
		Glass (BK7) n & 1.5159 & 1.5187 & 1.5267\\
		k & 1.1221e-8 & 6.9658e-9 & 1.1147e-8\\
		\hline
		Silver n & 0.15337 & 0.14461 & 0.13535\\
		k & 3.6381 & 3.1595 & 2.2550\\
		\hline
		Gold n & 0.22437 & 0.44553 & 1.4513\\
		k & 3.0219 & 2.2995 & 1.4513\\
		\hline
		Chromium n & 3.1940 & 2.7602 & 1.9166\\
		k & 4.2664 & 4.1903 & 3.7858\\
		\hline
		Iron n & 2.8920 & 2.9404 & 2.5075\\
		k & 3.3380 & 2.9279 & 2.7250\\
		\hline
		Calcite n & 1.6570 & 1.6616 & 1.6749\\
		k & 0.0000 & 0.0000 & 0.0000\\
		\hline
		Quartz n & 1.4577 & 1.4601 & 1.4667\\
		k & 0.0000 & 0.0000 & 0.0000\\
		\hline
		Cellulose n & 1.4692 & 1.4722 & 1.4810\\
		k & 0.0000 & 0.0000 & 0.0000\\
		\hline
	\end{tabular}
\end{document}