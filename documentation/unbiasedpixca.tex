\documentclass[]{article}
\usepackage{amsmath}
\usepackage{bbm}

%opening
\title{Unbiased Pixel Cache Tracing}
\author{Johannes Jendersie}

\begin{document}

\maketitle

\section{Pixel Cache Projection}

For a randomly chosen cache with position $\mathbbm{c}_{i}$ and a random point on the light path $\mathbbm{p}_{i}$ the unbiased light transport is
\begin{equation}
	w = \frac{f_r(\overrightarrow{\mathbbm{p}_{i-1}\mathbbm{p}_i}, \overrightarrow{\mathbbm{p}_i\mathbbm{c}_i})
	f_r(\overrightarrow{\mathbbm{p}_i\mathbbm{c}_i}, \overrightarrow{\mathbbm{c}_i, \mathbbm{c}_{i-1}})}
	{p_{cf}(\overrightarrow{\mathbbm{p}_i\mathbbm{c}_i})/p_c(\overrightarrow{\mathbbm{p}_i\mathbbm{c}_i})}
	\label{eq:transport0}
\end{equation}
where the denominator $p_{cf}/p_c$ is the probability for the connection direction under the distribution of caches.
It depends on the global spherical distribution of pixel caches $p_c(\mathbbm{d})$ with respect to $\mathbbm{p}$ and on the distribution of directions from the BRDF $p_f(\mathbbm{d}) = f_r(\overrightarrow{\mathbbm{p}_{i-1}\mathbbm{p}_i}, \mathbbm{d})$.
The common probability is
\begin{equation}
	p_{cf}(\mathbbm{d}) = \frac{p_c(\mathbbm{d})p_f(\mathbbm{d})}{\int_\Omega\int_\Omega p_c(\mathbbm{d})p_f(\mathbbm{d}) d\mathbbm{d} d\mathbbm{d}}
\end{equation}
because both destributions are independent.
The integrals normalize the distribution function to 1 and are the only reason why $w$ depends on $p_c$.
Putting this back into equation \ref{eq:transport0} gives
\begin{equation}
w = f_r(\overrightarrow{\mathbbm{p}_{i-1}\mathbbm{p}_i}, \overrightarrow{\mathbbm{p}_i\mathbbm{c}_i})
	f_r(\overrightarrow{\mathbbm{p}_i\mathbbm{c}_i}, \overrightarrow{\mathbbm{c}_i, \mathbbm{c}_{i-1}})
	\frac{\int_\Omega\int_\Omega p_c(\mathbbm{d})p_f(\mathbbm{d}) d\mathbbm{d} d\mathbbm{d}}
	{p_f(\overrightarrow{\mathbbm{p}_i\mathbbm{c}_i})}
\end{equation}

To solve the integral we would need to iterate over all existing caches which is unfeasible.
However, making assumptions to $p_f$ can remove this ugly term.

\subsection{Diffuse case}

Setting $p_f$ to an uniform distribution on the hemisphere gives the following quotient.
\begin{align}
p_f(\mathbbm{d}) &= \frac{1}{2\pi}\\
q &= \int_\frac{\Omega}{2}\int_\frac{\Omega}{2} p_c(\mathbbm{d}) d\mathbbm{d} d\mathbbm{d}\\
&= 2\pi p_c\left(\in\frac{\Omega}{2}\right)
\end{align}
Problem: we need to integrate over $\frac{\Omega}{2}$! $p_f$ is 0 otherwise.
I.e. the inner integral is the probability that the cache is in the positive hemisphere $p_c(\in\frac{\Omega}{2})$ which is constant.
The constant integrated over the hemisphere just adds another $2\pi$.
%With the assumption that $p_f=1/2\pi$ for the entire sphere we can go further.
%This is allowed because we discard the path anyway and the further weight is irrelevant.
%In that case the inner integral resolves to 1 (as this is a property of the density function $p_c$).
%\begin{align}
%q &= \int_\Omega 1 d\mathbbm{d}\\
%&= 4\pi
%\end{align}

\end{document}
